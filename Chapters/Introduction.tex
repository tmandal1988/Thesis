% Chapter Template

\chapter{Introduction} % Main chapter title

\label{Chapter1} % Change X to a consecutive number; for referencing this chapter elsewhere, use \ref{ChapterX}

\lhead{Chapter 1. \emph{Introduction}} % Change X to a consecutive number; this is for the header on each page - perhaps a shortened title

%----------------------------------------------------------------------------------------
%	SECTION 1
%----------------------------------------------------------------------------------------

\section{Motivation and Literature Review}
\doublespacing
\par \lettrine[lraise=0.3,loversize=1]{\textbf{A}}{}erospace engineers before Wright brothers knew that airfoil shaped cross section of a wing will provide sufficient lift for flying heavier than air machines. However Wright brothers' breakthrough invention of 3-axis aircraft control made it possible for a human pilot to fly and steer the plane safely\cite{birthofflightcontrol}. Even today pilot vehicle interaction receives lot of attention during design phase to develop a safe and efficient Pilot Vehicle System (PVS). However it is difficult to foresee all the possible way human pilot can interact with an aircraft due to human pilot's high degree of variability, non-linearity and adaptability.Complex nature of human operator demands a collective approach of analysis from the standpoint of control theory, psychology, neurology and biology of human sensors like eye, vestibule system etc.\cite{mcruer1965human}\cite{mcruer1969theory}\cite{hosman1999pilot}. If not properly designed and various handling issues regarding pilot vehicle interaction mitigated the PVS can lead to unstable conditions like Pilot Induced Oscillations (PIOs)\cite{McRuerPIO}. There has been several high profile PIO incidents\cite{SAABPIO}\cite{YF22} which emphasizes a need to better understand human pilot control characteristics as a part of overall PVS to reduce occurrence of such incidents.Military Standard Flying Qualities of Piloted Aircraft (MIL-STD 1797A)\cite{MLSTD} defines PIO to be “sustained or uncontrollable oscillations resulting from efforts of the pilot to control the aircraft”. Also it has been observed that recent advances in digital fly-by-wire technology with augmented controls has led to an increase in PIO events\cite{BoeingPIO}\cite{Norton}\cite{hoh1982bandwidth}\cite{hess1991technique}\cite{green1986design}. The increasing complexity of digital fly-by-wire technology makes it  further difficult to fully understand the stability properties of the PVS under various circumstances. However since fly-by-wire technology is becoming increasingly complex in order to meet the control requirements of highly maneuverable military jets and the safety requirements of commercial aircraft, further research is needed to predict their effects on the PIO susceptibility of the aircraft and to develop techniques to mitigate this problem.Many factors affect the susceptibility of a PVS to PIO and to properly study PIOs it is important to classify and organize them. In the literature there are three commonly accepted categories of PIOs as outlined in a 1997 summary report by a National Research Council (NRC) Committee on Effects of Aircraft-Pilot Coupling on Flight Safety\cite{mcruer1997aviation}, which are;\\
\textbf{Category 1:} This category of PIO mainly deals with oscillations with an underlying linear cause like excessive
time delay, phase loss etc. Pilot behavior can be modeled as quasi-linear and time stationary\cite{mcruer1997aviation} . Category 1 PIO is the least common and easiest to understand. Several criteria focusing on excessive phase loss and time delay have been developed and compared\cite{moorhouse2000flight}\cite{liu2010prediction}\cite{garteur1999evaluation}. Certain criteria are based on open loop analysis like Bandwidth/Pitch rate overshoot criteria\cite{mitchell2004recommended} , while criteria like Neal-Smith\cite{neal1970flight} and Smith-Geddes\cite{smith1978effects}\cite{smith1995observations}\cite{mitchell1998critical}\cite{smith1979handling} are closed loop analysis with an assumed pilot model.\\
\textbf{Category 2:} This category of PIO is characterized by nonlinear events which can be modeled as Quasi-linear
events such as actuator rate limiting or amplitude limiting, etc. These are the most common type of PIO observed\cite{mcruer1997aviation} .
Most of the PIOs associated with non-linear events were found to be cliff-like that is the pilot reported the onset of
the PIO as sudden and unexpected. Since control surface actuator rate limiting is a common non-linearity associated
with modern flight control systems\cite{klyde2004investigating}\cite{duda1997prediction}\cite{amato2000analysis}\cite{hanke1995handling}, most of the studies are focused on studying its influence on aircraft handling quality and PIO. Currently, Open Loop Onset Point\cite{duda1997prediction} (OLOP) developed by Holger Duda at DLR is the only commonly accepted criterion\cite{moorhouse2000flight}\cite{garteur1999evaluation} for Category 2 PIO resulting from rate limited actuator in fully rate saturated case.\\
\textbf{Category 3:} This category of PIO is caused by highly nonlinear events which involves transition in the control
element of the aircraft or the human pilot behavioral dynamics. The PIOs associated with this category are also cliff-
like. This is difficult to recognize and is relatively rare, but could be highly dangerous when it does occur\cite{mcruer1997aviation}. Due to its highly complex nature there are not many effective tools to study Category 3 PIO . Unmanned aerial test bed can be useful in getting more flight data for Category 3 PIO as they are extremely dangerous to be studied on manned flight.
\par Unmanned aerial test bed provides a high fidelity scaled down experimental setup which can generate data for different flight conditions at low costs.The expendable nature of the sub-scale test bed gives the flexibility to carry out dangerous and stressful flight experiments, such as sub-system failure during landing, without putting human lives in danger. It can also replicate digital fly-by-wire technology where pilot command is processed by on board computer before being sent to control surface actuators. The low operating costs and short turnaround time helps in generating actual flight data for PIOs in an efficient fashion\cite{mandal2013flight}. However pilot doesn't have motion cues from the flight but the unmanned aerial test bed can still be used to study single input single output human pilot compensatory control or pursuit behavior.In this case the input to human pilot is the visual information of Pitch angle from the actual flight.Other approach to study PIOs is to use simulations\cite{McRuerPIO} as they are free of the inconveniences of the actual flight tests, manned or unmanned. However simulations may not capture the full characteristics of the complexities and uncertainties involved with a real flight\cite{mcruer1997aviation}.Simulations can be used to obtain initial insights which can then be tested and expanded in actual sub-scale flight experiments.The acquired flight test data can then be used to improve the simulation and hence this way two methods can compliment each other.
\par As mentioned earlier human pilot behavior is characterized by highly dynamic and complex system and a higher level decision making process which depends on variety of external and internal factors\cite{stapleford1969experiments}\cite{kleinman1971control}\cite{johannsen1994theoretical}. The first systematic attempt in studying manual control of dynamic systems started during 1940s. The earlier work started with studying human as a controller of single variable, single display linear time-invariant systems. This however was not sufficient to generalize for much complex dynamic systems such as aircraft or a car\cite{land1994we}\cite{sentouh2009sensorimotor}.For  a  particular  flying 
condition consisting of simple task a human pilot can be satisfactorily represented as a quasi-linear system with a linear component consisting of corresponding gain, lead, lag and time delay and a non-linear remnant \cite{hess1990control}.This quasi linear model may not necessarily replicate the human pilot output exactly but is capable  of giving basic information about the behavioral properties of the human controller for control task in  systems like aircraft or car\cite{mcruer1967review}.Such behavioral properties include the system order, bounds on natural frequency and margins etc.\cite{shappell2012human}\cite{mcruer1969theory}\cite{hosman1999pilot}\cite{kleinman1971control}.The most widely 
used quasi-linear models for  human operator resulted from the work of Russell, Elkind, Mcruer and Krendel\cite{hess1990control} based on extensive experimental studies on different types of systems. The quasi-linear pilot model consists of parameters which represents the inherent limitations of human control, for example processing delay in human brain, delay in 
human neuromuscular system etc. For most single input single output problems this quasi-linear pilot models has been successfully used for human control analysis\cite{pool2009pilot}\cite{taylor1967comparison}\cite{ninz1982parametric}. At this point it is also worthwhile to mention that many researchers model human pilot as an optimal controller\cite{baron1969human}\cite{burnham1974identification}.Optimal control approach might be a good approximation if human controller is highly skilled and operating conditions are fairly fixed. However during flight lot of factors changes for example wind gust, aircraft dynamics (by failure injection)and human state of mind itself, all this factors may lead to sub-optimal performance. Hence This research work will apply quasi-linear pilot model to longitudinal control system of an aircraft where the input to the pilot model is the pitch  attitude  error and output of the  pilot  model  is  the  elevator  deflection.  Since  stability  along  pitch  axis  is  among  the  fundamental  stability requirement for an aircraft, the linear pilot/vehicle analysis can give rise to better pitch handling quality and can also be used to mitigate Pilot Induced Oscillations (PIO) in the longitudinal axis.

%-----------------------------------
%	SUBSECTION 1
%-----------------------------------
\section{Objective}
\doublespacing
\par The objective of this research is to study, quantify and derive the stability characteristics of the human pilot using sub-scale research aircraft.The work also focuses on setting up a general framework to carry out PVS experiments on a sub-scale research aircraft. This work analyzes pilot control behavior in the longitudinal axis of aircraft for pitch attitude control. The pilot is modeled as a single input single output quasi linear system with pitch error as input and elevator deflection as the output. Also Category I and Category II PIO experiments were conducted on the longitudinal axis of the sub-scale research aircraft. Category I PIO experiments were carried out by injecting delay in the pilot commands to the elevator. Category II PIO experiments were carried out by limiting the elevator actuator rate. The PIO experiments provides the experimental results of the condition that can cause PVS instability. The primary goal of this work is to be able to determine pilot model in real time and then use it to derive the stability margin of the PVS. This will then be used to obtain a  better prediction of PIO susceptibility in real time. Current pilot model estimation techniques focuses heavily on batch estimation techniques (for ex Least Squares, Maximum Likelihood)\cite{pool2009pilot}\cite{mandalpilot}\cite{van1979identification} where researchers look at the flight data from a particular flight phase and calculate the pilot model parameters in mean sense. This mean pilot model is then used to quantify the stability margins. Although it provides valuable insights into human pilot characteristics however it neglects the variability and adaptability nature of human pilot. In many case the batch estimation techniques might also suppress the variability in the pilot control data. The work aims to derive an estimation technique based on Kalman Filtering which estimates the pilot model in real time. The advantages are immediately apparent in terms of being able to capture the variable nature of human pilot control and at the same time having a real time tool to predict the susceptibility of aircraft to PIO.Previous real-time pilot  study involved wavelets\cite{thompson2001wavelet} to estimate time-varying frequency response  functions which is sensitive to  measurement noise\cite{zaal2011estimation} and does  not  concern with estimating pilot model parameters. Other techniques include windowed maximum likelihood\cite{zaal2011estimation} estimator which can be slow in determining changing pilot parameters.The KF  on  the  other  hand  is  easy  to  implement,  relatively  resistant  to  measurement  noise,  and  at  the  same  time  is  a powerful estimator. 

 Previous attempt at using Kalman filter\cite{schiess1975kalman} to estimate human pilot-model parameters has concluded that pilot model-parameters could be estimated if the magnitude of the process noise was sufficiently small. Also the study found that the remnant to pilot command ration was too small for their experiment. However if the low process noise assumption fails and initial conditions are drastically different than real values then the estimation may not give correct values of parameters. However at the same time there are multiple solutions to the parameters of the pilot model for which error between estimated pilot output and actual pilot output is low. All this mu

%-----------------------------------
%	SUBSECTION 2
%-----------------------------------
\section{Organization}
\doublespacing
\par The rest of this dissertation is organized as follows. Chapter 2 describes the details of the considered experimental flight testing platform,including an analysis and summary of the selected flight data. Chapter 3 presents background information on various parts of human pilot when modeled as a controller. Chapter 3 presents PIO analysis applied to the experimental platform. Chapter 4 presents the batch estimation applied to estimated human pilot model parameters for stead level flight and landing. Chapter 5 presents the real time estimation of pilot model parameters using KF.Chapter 6 presents the stability analysis and PIO prediction results.Finally,Chapter 7 summarizes and concludes this work and also discusses possible future direction.

