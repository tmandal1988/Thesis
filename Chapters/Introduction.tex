% Chapter Template

\chapter{Introduction} % Main chapter title

\label{Chapter1} % Change X to a consecutive number; for referencing this chapter elsewhere, use \ref{ChapterX}

\lhead{Chapter 1. \emph{Introduction}} % Change X to a consecutive number; this is for the header on each page - perhaps a shortened title

%----------------------------------------------------------------------------------------
%	SECTION 1
%----------------------------------------------------------------------------------------

\section{Motivation and Literature Review}

Aerospace engineers before Wright brothers knew that airfoil shaped cross section of a wing will provide sufficient lift for flying heavier than air machines. However Wright brothers' breakthrough invention of 3-axis aircraft control made it possible for a human pilot to fly and steer the plane safely\cite{birthofflightcontrol}. Even today pilot vehicle interaction receives lot of attention during design phase to develop a safe and efficient Pilot Vehicle System (PVS). However it is difficult to forsee all the possible way human pilot can interact with an aircraft due to human pilot's high degree of variability, non-linearity and adaptibility. If not properly desgined and various handling issues regarding pilot vehicle interaction mitigated the PVS can lead to unstable conditions like Pilot Induced Oscillations (PIOs)\cite{McRuerPIO}. There has been several high profile PIO incidents\cite{SAABPIO}\cite{YF22} which emphasizes a need to better understand human pilot control characteristics as a part of overall PVS to reduce occurance of such incidents.

%-----------------------------------
%	SUBSECTION 1
%-----------------------------------
\subsection{Objective}

Objective goes here

%-----------------------------------
%	SUBSECTION 2
%-----------------------------------

\subsection{Organization}
Organization goes here

